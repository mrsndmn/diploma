
% Тема:
% Программная реализация алгоритмического обеспечения
% для решения задачи нейро-нечеткого моделирования
% социально-экономических систем по данным открытых источников.

\chapter{Анализ проблематики }
\label{chapter1}

\begin{annotation}
	В данной главе приводится анализ предметной области.
	Проводится сравнительный анализ методов, с помощью которых
	можно решать задачу моделирование сложных систем.
	В конце раздела формулируются цели и задачи работы.
\end{annotation}


\section{Анализ предметной области}

Социально-экономические системы обычно зависят от большого количества параметров.
Эти параметры могут неявно влиять друг на друга. Для описания состояния такой системы
можно использовать временные ряды. Для того, чтобы изучить такую систему, нужно построить ее модель.

Задача моделирования сложных систем ~- это определение возможных параметров
системы и значений этих параметров. После того, как параметры модели были определены,
можно исследовать поведение системы по полученной модели. Одним из возможных
применений полученной модели может быть прогнозирование поведения модели в будущем.
Также модель может помочь проверить гипотезы, которые выдвигают эксперты.
Кроме того, эксперты могут исследовать причинно-следственные связи на основе моделирования.

Модель социально-экономической системы может быть использована
для прогнозирования будущих значений параметров системы. Прогноз строится на основании
истории одного или нескольких временных рядов, которые соответствуют историческим данным.

\begin{definition}
	(Временной ряд)
	Временным рядом называются последовательно измеренные через некоторые промежутки времени данные \ref{voron}.
	% todo link to http://www.machinelearning.ru/wiki/images/archive/c/cb/20160412121749!voron-ml-forecasting-slides.pdf
\end{definition}


Основные явления в эконометрических временных рядах \cite{voron}:
\begin{itemize}
	\item тренды ~= очищенная от случайностей основная тенденция временного ряда
	\item сезонности ~= периодические отклонения от тренда
	\item разладки ~= резкое изменение свойств наблюдаемого ряда
\end{itemize}

Сезонности тоже можно разделить на несколько типов:

\begin{itemize}
	\item аддитивная сезонность
	\item мультипликативная сезонность
\end{itemize}

Так как в системе может быть много параметров (и, соответственно, временных рядов),
при прогнозировании модель может вычислять будущие значения параметров разными способами:

\begin{itemize}
	\item предсказывать новые значения временного ряда на основе его прошлых значений
	\item предсказывать новые значения временного ряда на основе предыдущих значений
	нескольких параметров, которые могут повлиять на данный параметр
	\item предсказывать сразу несколько временных рядов на основании их предыдущих значений
\end{itemize}

Эти способы перечислены в порядке усложнения модели. Сложность модели влияет на
качество ее предсказаний. Однако выбор слишком сложной модели может повлечь
за собой ее переобучение. Переобучение ведет к тому, что модель теряет способность
к обобщению. Кроме того, сложные модели нужно дольше обучать. И нужно больше данных
для ее качественного обучения. Сложные модели сложнее интерпретировать.

Основные виды моделей для предсказаний временных рядов \cite{voron}:
\begin{itemize}
	\item Aвторегрессионные модели ~= значения временного ряда в данный момент линейно зависят от предыдущих значений этого же ряда
	\item Адаптивные модели ~= самонастраивающиеся модели, которые способны отражать изменяющиеся во времени условия
	\item Нейросетевые модели
\end{itemize}

% todo адаптивные модели, на самом деле, они являются авторегрессионными?

\section{Сравнительный анализ методов прогнозирования временных рядов}

\textbf{Адаптивные модели} хорошо работают на большом количестве временных рядов.
В случаях, когда необходимо прогноз нужно получить быстро, а учитывать
нужно большое количество факторов. Суть адаптивных
методов заключается в том, что на каждой итерации, после того, как стали
известны новые данные, параметры модели обновляются. Параметры модели вычисляются
на основе исторических данных. Однако такие модели не могут описать сложные
зависимости.

Простота адаптивных методов компенсируется селективными и композиционными моделями.
Можно обучить несколько разных моделей и в зависимости от качества предсказаний на пошедшие
моменты времени для прогнозирования можно использовать ту модель, качество предсказаний
которой выше -~ в этом заключается идея селективных моделей. В композиционных моделях,
результирующим предсказанием выступает взвешенная сумма предсказаний моделей. Вес каждой модели
тоже адаптивный: вычисляются на основании ошибки данной модели.

Обычно адаптивные модели используются для краткосрочного моделирования.
Такие модели при прогнозировании временных рядов учитывают значения только одного временного ряда.
То, что на один временной ряд может влиять значения других временных рядов никак не учитывается.

Преимуществом адаптивных моделей можно считать их простоту и то, что они могут подстраиваться
под изменяющиеся параметры временного ряда.

Простейшим примером адаптивной модели является модель экспоненциального сглаживания \ref{exp_smoothing}.
Наблюдения учитываются с убывающими весами. % \ref{exp_smoothing_decreasing_weights}
Чем больше значение $ \alpha $, тем больше сглаживается результат вычислений модели.

\def\figurename{Формула}
\begin{figure}
	\centering
	$ \hat{y_{t+1}} = \alpha y_t + (1 - \alpha)\hat{y_t} $,
	\caption{
		Модель экспоненциального сглаживания.
		$ \alpha $ -~ параметр сглаживания
	}
	\label{img:exp_smoothing}
\end{figure}
\def\figurename{Рис.}


\textbf{Авторегрессионные модели} могут учитывать влияние большого количества параметров
временного ряда. Но чем больше параметров будет иметь модель, тем больше вычислений
потребуется для оценки значений этих параметров.

ARMA -~ модель авторегрессии скользящего среднего. Используется для прогнозирования
стационарных временных рядов.
ARIMA -~ это интегрированная модель авторегрессии. Позволяет объяснить тренд во временном
ряде за счет интегрирования: после того, как посчитаны разности исходного временного ряда,
пелается предположение, что эти разности -~ это стационарный ряд и предсказывается поведение
интегрированного ряда.
ARIMAX -~ модификация ARIMA, в которой в модели регрессии могут учитываться не только предыдущие значения
текущего временного ряда, но и другие экзогенные переменные.
SARIMAX -~ ARIMAX с учетом сезонной компоненты. Для учета сезонной компоненты
берутся параметры авторегрессии, интегрирования и скользящего среднего с отставанием
определенным периодом сезонности.


В \textbf{нейросетевых моделях} тоже можно учесть влияние одного параметра системы на другие.
Суть данных методов похожа на предыдущие: сначала настраиваются начальные
значения параметров модели, а потом эти параметры обновляются, если добавляются новые данные.
Основным отличием можно считать то, что в предсказании одного временного ряда можно
учитывать значения других временных рядов в разные промежутки времени.

Нейросетевые методы можно разделить на 2 вида:
\begin{itemize}
	\item сети прямого распространения
	\item рекуррентные
\end{itemize}

Сети прямого распространения не подразумевают какого-либо состояния.
Поэтому передать информацию о предыдущем шаге итерации можно только явно:
предыдущие значения временного ряда должны передаваться на вход модели.
К сожалению, при таком подходе, количество параметров сети возрастает очень быстро
при увеличении количества предыдущих значений временного ряда. Это сильно усложняет модель,
она легко переобучается. И не может адекватно предсказать новые значения для временного ряда,
потому что теряет способность к обобщению.

Рекуррентные сети наравне с параметрами имеют внутреннее состояние, которое обновляется на каждой
итерации. Это означает, что предсказание будущих значений рядов зависит от "контекста",
то есть от предыдущих вычислений. Данные сети также могут учитывать влияние других
временных рядов на текущий даже с учетом нескольких предыдущих значений.

Но при описании влияния нескольких временных рядов в нейросетевых моделях
нельзя описать, на какой именно ряд влияет другой ряд. Такие знания могут быть у
экспертов. И потенциально, эти знания могли бы упростить модель и уменьшить количество
параметров в ней.

% http://www.machinelearning.ru/wiki/images/archive/c/cb/20160412121749!voron-ml-forecasting-slides.pdf
% todo Как огромное кол-во рядов, после того, как они были предсказаны, потом выливается в предсказание продаж?
% потом просто полносвязной по всем рядам вычисляют значение целевого временного ряда?
% деревья решений?


\section{Выводы}

\begin{enumerate}
	\item Адаптивные модели хорошо работают на большом количестве временных рядов,
	но не учитывают влияние других рядов на текущий. Они хорошо предсказывают простые зависимости.
	И имеют возможность адаптироваться к новому уровню процесса.
	\item Модификации регрессионных моделей могут позволить описать значительную
	часть информации о временном ряде. Они более интерпретируемы, чем нейросетевые методы, более просты.
	\item Сети прямого распространения плохо подходят для предсказания временных рядов.
	\item Рекуррентные сети хорошо подходят для предсказания временных рядов, но
	с помощью них сложно представить знания о зависимостях между временными рядами.
\end{enumerate}


\section{Постановка задачи работы}


В рассмотренных методах не хватает возможности структурировать экспертные знания.
Для решения этой проблемы могут быть использованы когнитивные карты.
Кроме этого, когнитивные карт разных экспертов можно объединять. Обычно
объединенная карта представляет данные более объективно, потому что возможные
ошибки одного эксперта могут быть компенсированы знаниями другого эксперта.

Целью данной работы является создание нечеткой когнитивной карты для моделирования
социально-экономических систем с использованием авторегрессионных и нейросетевых моделей.

Задачи данной работы:
\begin{itemize}
	\item Разработать алгоритм для интеграции нейросетевых моделей с когнитвными картами.
	\item Сравнить разные модели предсказания временных рядов между собой.
	\item Спроектировать и реализовать модель для когнитивного картирования.
	\item Протестировать полученную модель.
\end{itemize}

