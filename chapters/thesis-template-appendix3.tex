
\chapter{Приложение 3. Описание исследуемого набора данных}


\begin{table}[h]
    \centering
    \begin{tabular}{|l|r|}
		\hline
			Характеристика	& Значение \\
		\hline
			Занимаемый объем оперативной памяти & 3.7MB      \\
			Количество строчек в наборе данных  & 17217      \\
            Количество столбцов                 & 18         \\
			Минимальное значение поля "date"    & 2011-01-29 \\
            Максимальное значение поля "date"   & 2016-04-24 \\
		\hline
	\end{tabular}
	\caption{ Количественные характеристики исследуемого набора данных }
	\label{tbl:sales_dataset_description_quantitive}
\end{table}


\begin{table}[h]
    \centering
    \begin{tabular}{|l|l||l|}
		\hline
			Название столбца	& Тип данных				& Описание \\
		\hline
			date				& Дата 			& "2011-01-29", "2013-10-23			\\
			cat\_id\_FOODS		& Число			& 1, если товар в категории "Еда", иначе 0 \\
			cat\_id\_HOBBIES		& Число		& 1, если товар в категории "Товары для хобби", иначе 0 \\
			cat\_id\_HOUSEHOLD	& Число			& 1, если товар в категории "Товары для дома", иначе 0 \\
			store\_id\_TX\_1		& Число		& 1, если товар был в магазине "TX\_1", иначе 0  \\
			store\_id\_TX\_2		& Число		& 1, если товар был в магазине "TX\_2", иначе 0  \\
			store\_id\_TX\_3		& Число		& 1, если товар был в магазине "TX\_3", иначе 0  \\
			weekday\_Monday			& Число  	& 1, если дата --- понедельник, иначе 0		\\
			weekday\_Tuesday		& Число  	& 1, если дата --- вторник, иначе 0		\\
			weekday\_Wednesday		& Число  	& 1, если дата --- среда, иначе 0		\\
			weekday\_Thursday		& Число  	& 1, если дата --- четверг, иначе 0		\\
			weekday\_Friday			& Число  	& 1, если дата --- пятница, иначе 0		\\
			weekday\_Saturday		& Число  	& 1, если дата --- суббота, иначе 0		\\
			weekday\_Sunday			& Число  	& 1, если дата --- воскресенье, иначе 0	\\
			event\_type\_Cultural	& Число 	& 1, если дата --- культурный  праздник, иначе 0 \\
			event\_type\_National	& Число 	& 1, если дата --- национальный  праздник, иначе 0 \\
			event\_type\_Religious	& Число 	& 1, если дата --- религиозный  праздник, иначе 0 \\
			event\_type\_Sporting	& Число 	& 1, если дата --- спортивный праздник, иначе 0 \\
		\hline
	\end{tabular}
	\caption{ Описание набора данных по количеству продаж товаров разных категорий. }
	\label{tbl:sales_dataset_description}
\end{table}

