\chapter{Разработка моделей и алгоритмов}

% В этой главе описываются разработанные/модифицированные модели/методы/
% алгоритмы, или/и описывается применение известных стандартных методов. Также,
% в конце главы обычно приводится общая архитектура программной системы,
% вытекающая из описанной теории. Приведенные ниже заголовки подразделов так же
% весьма примерные и сильно зависят от особенностей конкретной работы.


\begin{annotation}
	В данной главе описываются алгоритмы работы нечетких когнитивных карт.
	Описывается модифицированная модель когнитивных карт с использованием
	рекуррентных нейросетей и сетей прямого распространения.
	Производится выбор метрик оценки качества работы системы.
\end{annotation}

\section{Описание алгоритма работы нечетких когнитивных карт}

\begin{annotation}
	В данном разделе рассматриваются принципы работы нечетких
	когнитивных карт. Описываются основные понятия и формулы
	для работы с нечеткими когнитивными картами.
	Рассматриваются методы прогнозирования и исследования сложных систем
	с помощью нечетких когнитивных карт.
\end{annotation}


% Стандартная формула для пересчета значений концептов нечетких когнитивных карт представлена на
% \ref{img:concepts_recalc}.

% $ A_i(t) $ - значение концепта $A_i$ на шаге $t$.

% $ p(x) $ - функция активации \ref{img:sigmiog_actiovation}. Параметр $m$ определяет, на сколько сигмоида будет похожа на пороговую функицю.

% \def\figurename{Формула}
% \begin{figure}[t]
% 	\centering
% 	$ A_j(t+1) = p( A_j(t) + \sum_{i = 1, i \neq j}^{n} w_{ij} A_i(t) ) $,
% 	\caption{Стандартная формула пересчета значений концептов}
% 	\label{img:concepts_recalc}
% \end{figure}
% \noindent

% \begin{figure}[t]
% 	\centering
% 	$ p(x) = \frac {1} {1+ e^{-mx} } $,
% 	\caption{Сигмоидальная функция активации}
% 	\label{img:sigmiog_actiovation}
% \end{figure}
% \def\figurename{Рис.}

% В зависимости от задачи и данных, с которыми работает эксперт могут использоваться и другие
% функции активации.

% После того, как эксперт построил карту, он может приступить к моделированию поведения
% системы и изучением причинно-следственной связи. Таким образом, эксперт может посмотреть,
% как изменение одного концепта может повлиять на поведение системы в целом.

% Если значения концептов карты расходятся, эксперту нужно или попробовать изменить функцию
% активации или изменить значения весов концептов. Возможно, карта несбалансированна из-за
% того, что в ней отсутствует еще один компонент, который вносит значительный вклад в поведение системы.

% Пересчет значений концептов проходит итеративно. Имея исторические данные эксперт может
% подобрать значения весов таким образом, чтобы смоделированное поведение системы как можно меньше отличалось
% от экспериментальных данных. В случае классических когнитивных карт эксперту нужно это делать вручную.

% После того, как эксперт оптимизировал значения весов, можно приступить к изучению причинно-следственных
% связей. Также эксперт может получить прогнозировать значение определенных концептов, если проведет еще несколько
% итераций пересчета значений концептов.

\section{Интеграция когнитивных карт и нейросетей для предсказания временных рядов}

\begin{annotation}
	В данном разделе рассматриваются модификации нечетких когнитивных карт
	с использованием нейросетей.
	Приводятся алгоритмы вычислений концептов для данной модификации. Описываются
	как, можно применить нейросетевые методы для оптимизации весов нечетких когнитивных карт.
\end{annotation}



% раньше было в 1 главе
% -=================================

% \section{Интеграция когнитивных карт и нейросетей для предсказания временных рядов}


% Когнитивная карта – схема причинно-следственных связей в системе.
% Когнитивная карта представляет из себя ориентированный граф.
% Вершины этого графа – концепты, а ребра – причинно-следственные связи между соответствующими концептами.
% Когнитивные карты строятся для того, чтобы понять и проанализировать структуру сложной системы.
% Каждое ребро когнитивной карты имеет свой вес, который характеризует степень влияния одного концепта на другой.

% Обычно при прогнозировании с использованием нечетких когнитивных карт существует один целевой концепт,
% значение которого необходимо спрогнозировать.

% Нейро-нечеткая система - это система, которая включает в себя методы нейронных сетей и методы нечеткой логики.

% Поведение объекта в сложных системах прогнозировать непросто.
% Часто необходимо знать не только результат прогноза,
% но и причинно-следственные связи, которыми был обусловлен данный прогноз
% \cite{osoba2019dags} \cite{efficient_fcms}.
% Для выявления таких причинно-следственных связей используются нечеткие когнитивные карты.
% Они позволяют качественно оценить влияние разных компонент системы на другие компоненты.
% Однако, оригинальный алгоритм для пересчета значений концептов
% не может описать сложные зависимости между отдельными компонентами.
% Поэтому для того, чтобы увеличить мощность множества функций,
% которые могут быть использованы для описания влияния концептов карты друг на друга
% можно использовать нейросетевые методы.
% В таком случае когнитивная карта все еще хранит данные о причинно-следственных связях.
% В чисто нейросетевых подходах установить причинно-следственные связи намного сложнее.
% Другие системы, например, деревья решений, наоборот, очень легко интерпретируются, легко обучаются,
% но имеют гораздо более слабую способность к обобщению.


% Идея нечетких когнитивных карт, предложенных Коско \cite{kosko1986fuzzy} имеет следующие преимущества:

% \begin{itemize}
% 	\item Для проведения вычислений требуются небольшие вычислительные мощности.
% 	\item Простота в использовании.
% 	\item Можно легко добавлять и удалять концепты.
% \end{itemize}

% Однако есть и недостатки:

% \begin{itemize}
% 	\item Веса линейные. Невозможно представить нелинейную зависимость между концептами.
% 	\item Значение концепта зависит только от значений связанных концептов карты только на предыдущей итерации.
% 	Невозможно определить связанные во времени концепты более дольше, чем на одну итерацию.
% 	\item Для работы карты ее нужно настраивать и составлять экспертно.
% \end{itemize}

% Эти недостатки можно считать весьма существенными.
% Поэтому идея нечетких когнитивных карт развивалась до идеи TTR NFCM (Thee Term Relation Neural Fuzzy Cognotive Map)
% \cite{threeTermNfcm}.

% В отличие от классических нечетких карт TTR NFCM позволяет
% определять нелинейные функции для весов и при перерасчете значений концептов может учитывать то, как
% этот концепт менялся во времени. Возможность использовать нелинейные веса достигается с помощью многослойных прецептронов.
% А влияние предыдущих значений концептов определяется размером временного окна, за период которого проводится перерасчет
% значений концептов.
% Кроме того, в TTR NFCM используется три составляющих для минимизации ошибки при обучении:
% \begin{itemize}
% 	\item Пропорциональная составляющая.
% 	\item Интегрирующая составляющая.
% 	\item Дифференцирующая составляющая.
% \end{itemize}
% \noindent данный принцип был позаимствован у PID контроллеров.

% Однако по сравнению с классическими нечеткими когнитивными картами у TTR NFCM можно выделить следующие недостатки:
% \begin{itemize}
% 	\item Больше вычислительная сложность.
% 	\item Сложнее реализация системы.
% \end{itemize}

% Тем не менее TTR NFCM не увеличивает количество параметров, которые должны быть определены экспертно.
% То есть работа с такой когнитивной картой будет такой же простой, как и с классической FCM.

% Существуют и другие модификации нечетких когнитивных карт, например,
% Fuzzy Grey Cognitive Maps, the Intuitionistic Fuzzy Cognitive Maps, но они увеличивают количество параметров,
% которые должны быть определены экспертно.


% -=================================

\section{Выбор метрик для оценки качества работы алгоритмов}

\begin{annotation}
	В данном разделе производится выбор и описание метрик для оценки
	качества работы нейро-нечеткой системы. Описывается смысл приведенных метрик.
	Приводятся расчетные формулы для выбранных метрик.
\end{annotation}

% В качестве критерия качества результата работы системы можно взять
% аккумулированную квадратичную ошибку для каждого концепта на каждом шаге итерации.
% \ref{img:loss_func}

% Это значение можно считать для каждого концепта. Но для простоты интерпретации результатов,
% можно и сложить и изменить ошибку для всех концептов на определенном шаге итерации.

% \def\figurename{Формула}
% \begin{figure}[t]
% 	\centering
% 	$ Err(t) = \sum_{i = 0}^{n} ( y_i - \hat{y_i} )^{2} $,
% 	\caption{Критерий качества работы карты}
% 	\label{img:loss_func}
% \end{figure}
% \def\figurename{Рис.}



\section{Выводы}

В данной главе была рассмотрена модель когнитивных карт.
Была разработана модель когнитивных карт
с использованием нейросетей (рекуррентных и прямого распространения).
Были выбраны методы оценки качества работы разрабатываемой системы.