\chapter{Результаты проектирования}

\begin{annotation}
	В данной главе разрабатываются функциональные и пользовательские требования к веб-приложению.
	Проектируется общая архитектура приложения для автоматического когнитивного картирования.
	Описаны этапы обработки данных для автоматического когнитивного картирования.
	Описывается архитектура приложения.
\end{annotation}


% ================ из 2 главы


% \section{Описание этапов обработки данных для когнитивного картирования}

% % \begin{annotation}
% % 	В данном разделе описываются методы обработки данных для автоматического когнитивного
% % 	картирования. Описывается, как можно интерпретировать результат разрабатываемой системы
% % 	и как обработка данных может повлиять на результат вычислений.
% % 	Описываются теоретические принципы работы разрабатываемой системы.
% % \end{annotation}

% Данные в систему могут поступать из разных источников. Так как данные из разных источников
% могут быть в разных форматах, в модулях, в которых производится предобработка данных
% они приводятся к единому виду и нормализуются. После этого собранные данные должны подвергнуться
% анализу. В процессе этого анализа наша система обогащает данные метками или тегами, задает каждому
% объекту вес. Часть меток может браться из первоисточника, если первоисточник предоставляет такие данные.

% Таким образом перед тем, как эксперт может начать работать с данными, они проходят следующие этапы:

% \begin{itemize}
% 	\item скачивание
% 	\item нормализация
% 	\item обогащение
% \end{itemize}

% Так как полученных данных может оказаться слишком много, для работы эксперт может ограничить количество
% концептов различного типа. И уже на их основе разрабатывать карту.

% Под нормализацией данных подразумевается то, что слова в текстах должны быть приведены к единообразному виду.
% Для этого можно использовать стемминг и лемматизацию. Так как данных в системе может быть много, а лемматизация
% на порядок дольше \cite{vallbe2007stemming}, чем стемминг, в данной работе будет исползоваться стемминг.

% \begin{definition}
% Стемминг - это процесс нахождения основы слова для заданного исходного слова.
% Основа слова не обязательно совпадает с морфологическим корнем слова.
% \end{definition}

% \begin{definition}
% Лемматизация - процесс приведения словоформы к лемме — её нормальной (словарной) форме.
% \end{definition}

% Лемматизация требует более строго разбора слова и достаточно требовательна к вычислительным ресурсам.

% =================


\section{Разработка функциональных требований к системе}

Так как разрабатываемый модуль предназначен для решения задачи, которая
требует интенсивные вычисления, система должна быть реализована
с возможностью оптимизации данных вычислений. Распараллеливать вычисления
можно с помощью видеокарт. Это можно сделать с использованием фреймворка pytorch.

Для проверки качества кода и его работоспособности, должны быть написаны тесты.
Автотесты помогают быстрее находить ошибки и регрессии в функциональности программы.

Эксперт должен иметь возможность задать произвольные концепты, описать
взаимосвязь между концептами явно или неявно, с помощью обученной карты.
Для того, чтобы обучить карту, эксперт должен иметь возможность загрузить
в нее исторические данные.

Эксперт должен иметь возможность построить графики зависимостей концептов
как по историческим данным, так и по предсказанным данным.

Таким образом, функциональные требования:
\begin{itemize}
	\item Возможность распараллеливания вычислений, с использованием фреймворка pytorch.
	\item Система непрерывной интеграции и тесты.
	\item Поддержка заданных экспертом зависимостей между концептами.
	\item Поддержка зависимостей между концептами, вычисляемых с помощью LSTM.
	\item Возможность визуально представить данные о карте: об истории обучения, качестве модели.
\end{itemize}

% todo как идея, можно прикрутить мониторинг того, как обучается модель

\section{Проектирование интерфейса модуля}

Работа с модулем должна состоять из следующих шагов:
\begin{itemize}
	\item Эксперт описывает карту и загружает в нее исходные данные.
	\item Карта обучается.
	\item Выполняется предсказание.
	\item Карта модифицируется, дополняется.
	\item И т д
\end{itemize}

Описание концептов:

Концепты для автоматического обучения:

Концепты для ручного обучения:

Обучение:

Вычисление карты:

Модификация карты:

Повторное обучение:
(можно переобучать не все, а только часть концептов)


\section{Выводы}
