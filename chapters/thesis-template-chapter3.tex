\chapter{Результаты проектирования}

\begin{annotation}
	В данной главе разрабатываются функциональные и пользовательские требования к веб-приложению.
	Проектируется общая архитектура приложения для автоматического когнитивного картирования.
	Описаны этапы обработки данных для автоматического когнитивного картирования.
	Описывается архитектура приложения.
\end{annotation}


\section{Разработка функциональных требований к системе}

Так как разрабатываемый модуль предназначен для решения задачи, которая
требует интенсивные вычисления, система должна быть реализована
с возможностью оптимизации данных вычислений. Распараллеливать вычисления
можно с помощью видеокарт. Это можно сделать с использованием фреймворка pytorch.

Для проверки качества кода и его работоспособности, должны быть написаны тесты.
Автотесты помогают быстрее находить ошибки и регрессии в функциональности программы.

Эксперт должен иметь возможность задать произвольные концепты, описать
взаимосвязь между концептами явно или неявно, с помощью обученной карты.
Для того, чтобы обучить карту, эксперт должен иметь возможность загрузить
в нее исторические данные.

Эксперт должен иметь возможность построить графики зависимостей концептов
как по историческим данным, так и по предсказанным данным.

Таким образом, функциональные требования:
\begin{itemize}
	\item Возможность распараллеливания вычислений, с использованием фреймворка pytorch.
	\item Система непрерывной интеграции и автоматизированное тестирование.
	\item Поддержка заданных экспертом зависимостей между концептами.
	\item Поддержка зависимостей между концептами, вычисляемых с помощью LSTM.
	\item Возможность визуально представить данные о карте: об истории обучения, качестве модели.
	\item Модуль должен быть расширяемым. Нужна возможность определить пользовательскую функцию для постобработки данных, сгенерированных картой.
\end{itemize}

% todo как идея, можно прикрутить мониторинг того, как обучается модель

\section{Проектирование интерфейса модуля}

Работа с модулем должна состоять из следующих шагов:
\begin{itemize}
	\item Эксперт описывает карту и загружает в нее исходные данные.
	\item Карта обучается.
	\item Выполняется предсказание.
	\item Карта модифицируется, дополняется.
	\item И т д
\end{itemize}

Описание концептов:

Концепты для автоматического обучения:

Концепты для ручного обучения:

Обучение:

Вычисление карты:

Модификация карты:

Повторное обучение:
(можно переобучать не все, а только часть концептов)


\section{Выводы}
