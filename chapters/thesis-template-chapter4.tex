\chapter{Реализация и экспериментальная проверка}

\begin{annotation}
	В данной главе приводятся детали разработки и экспериментальной проверки
	разработанной системы. Описан выбор инструментов, используемых для реализации
	программного обеспечения. Приводится выбор инфраструктурных решений,
	с учетом поставленных ранее системных и функциональных требований.
	Описаны состав и структура реализованного программного обеспечения.
\end{annotation}


% todo можно привести таблицу AKAIKE с разными гиперпараметрами для


\section{Выбор инструментальных средств}
\begin{annotation}
	В данном разделе обосновывается выбор инструментальных средств с учетом выдвинутых в 3
	главе требований к системе. Рассматриваются инструменты для обработки и хранения данных,
	для разработки backend и frontend части приложения. Рассматриваются инструменты для работы
	с нейросетями и графами для реализации компонента, связанного с вычислением нейро-нечеткой системы.
\end{annotation}



\section{Состав и структура реализованного программного обеспечения}
\begin{annotation}
	В данном разделе рассматривается состав ия структура реализованного программного обеспечения.
	Приводятся характеристики разработанного программного продукта, возможности конфигурации отдельных
	компонент системы. Описано назначение исполняемых файлов, описаны требования к системному окружению.
\end{annotation}

Реализованное по --- это модуль на python.
Нужно прикрутить автотесты, coverage, линтеры

\section{Анализ данных}

Набор данных, на которых будет проводиться тестирование разработанной системы
представляет из себя несколько $ csv $ файлов: файл с количеством продаж
продуктов разных категорий в 3 магазинах за 5 лет начиная с 29.01.2011
и файл с описанием праздников, которые выпадают на каждый день.

Требуется предсказать количество продаж на 30 дне вперед.

Для начала можно провести качественный визуальный анализ данных.

Общее количество продаж можно оценить на графике \ref{img:all_sales}.
Из этого графика видно, что до середины 2012 года количество проданных
товаров увеличивалось. Но после тренд пропал. В данных можно наблюдать
как годичные, так и месячные и даже недельные сезонности. Каждый год
в Рождество количество проданных товаров равно 0, потому что магазин не работает.
По выходным количество проданных товаров больше, чем в будние дни.

Количество проданных товаров по каждому магазину представлено на следующем графике \ref{img:sales_by_store}.
В количестве проданных товаров отдельно по каждому магазину нет значительных отличий.
Все магазины продают примерно одинаковое количество товаров.

Количество продаж по категориям для каждого магазина на графику \ref{img:sales_by_store_by_cat}
Во всех трех магазинах количество продаж товаров в категории "еда" больше всего,
а товаров для дома меньше всего. Из данного графика можно заметить, что сезонность
присутствует в графике по продаже товаров в категории "еда" и для товаров для дома, но
для товаров для хобби сезонности не наблюдаются.

Рассмотрим графики автокорреляции и частичной автокорреляции для общего количества продаж
\ref{img:any_autocorrelation}.
Для лагов 1, 2, 3, 6, 7 присутствует статически значимая частичная автокорреляция.
Это может быть использовано при построении модели SARIMAX.
На графики автокорреляции видны синусоидельные колебания. Это свидетельствует и подтверждает,
что в данных есть сезонность.

\def\figurename{Рис}
\begin{figure}[t]
	\centering
	\includegraphics[width=0.9\columnwidth]{./img/all_sales.png}
	\caption{Количество продаж}
	\label{img:all_sales}
\end{figure}


\def\figurename{Рис}
\begin{figure}[t]
	\centering
	\includegraphics[width=0.25\columnwidth]{./img/store_tx1_total.png}
	\includegraphics[width=0.25\columnwidth]{./img/store_tx2_total.png}
	\includegraphics[width=0.25\columnwidth]{./img/store_tx3_total.png}
	\caption{Общее количество продаж по каждому магазину}
	\label{img:sales_by_store}
\end{figure}

\def\figurename{Рис}
\begin{figure}[t]
	\centering
	\includegraphics[width=0.25\columnwidth]{./img/store_tx1_by_cats.png}
	\includegraphics[width=0.25\columnwidth]{./img/store_tx2_by_cats.png}
	\includegraphics[width=0.25\columnwidth]{./img/store_tx3_by_cats.png}
	\caption{Количество продаж товаров определенной категории для каждого магазина}
	\label{img:sales_by_store_by_cat}
\end{figure}

\def\figurename{Рис}
\begin{figure}[t]
	\centering
	\includegraphics[width=0.4\columnwidth]{./img/sales_autocorrelation.png}
	\includegraphics[width=0.4\columnwidth]{./img/sales_partial_autocorrelation.png}
	\caption{Автокорреляция и частичная автокорреляция для общего количества продаж}
	\label{img:any_autocorrelation}
\end{figure}



\section{Построение модели SARIMAX}

Для оптимального подбора гиперпараметров был проведен
визуальный анализ исходных данных, а так же проведена
кросс-валидация с поиском модели с минимальным значением критерия Акаике \cite{akaike}.
Оптимизация параметров на полном наборе данных занимает примерно 10 секунд при 8 параметрах модели.

Оптимальными гиперпараметрами оказались:
$ p = 2, d = 0, q = 3, P = 0, D = 1, Q = 1, S = 7 $.
Сезонность недельная  -~ это логично, что в текущий день
товаров будет продано столько же, как неделю назад.

На графике остатков \ref{img:arimax_resid} все еще прослеживается
годичная сезонность и выбросы. Выбросы -~ предсказать возможно, если
добавить экзогенные параметры в виде индикатора праздников.

\def\figurename{Рис}
\begin{figure}[t]
	\centering
	\includegraphics[width=0.5\columnwidth]{./img/arimax_resid.png}
	\caption{График остатков модели ARIMAX(2, 0, 3)x(0, 1, 1, 7)}
	\label{img:arimax_resid}
\end{figure}

В качестве экзогенных параметров рассмотрим индикатор дня недели.
В исходные данные добавится 7 колонок. Каждому дню недели
будет соответствовать одна колонка. И еще одним экзогенным параметром
возьмем индикатор праздника -- Рождества.
Экзогенные параметры можно вычислить заранее, так как наперед можем
поставить в соответствие индикатор дня недели определенному календарному дню
и индикатор Рождества.
Оптимизация параметров модели с 8 экзогенными переменными потребовало 20 секунд.

На графике остатков модели с экзогенными переменными \ref{img:arimax_resid_with_xmas} видно, что удалось избавиться
от большого значения остатков на Рождество. Это получилось сделать благодаря
столбцу-индикатору Рождественских праздников.

\def\figurename{Рис}
\begin{figure}[t]
	\centering
	\includegraphics[width=0.5\columnwidth]{./img/arimax_resid_with_xmas.png}
	\caption{График остатков модели ARIMAX(2, 0, 3)x(0, 1, 1, 7). Экзогенные переменные: индикаторы дней недели и Рождества}
	\label{img:arimax_resid_with_xmas}
\end{figure}

Но если проанализировать значения коэффициентов обученной модели \ref{tbl:arimax_coeffs_exogen},
можно заметить, что индикаторы дня недели слишком слабо влияют на предсказание модели.
Значения коэффициентов очень маленькие, вероятность того, что эти параметры будут иметь
значение больше критического равно единице, дисперсия большая. Из этого можно сделать вывод,
что использование этих экзогенных переменных неоправданно. Но зато значение коэффициента
для индикатора Рождества ($ xmas $) большое. Этот экзогенный параметр
действительно объясняет какую-то полезную часть данных.

\begin{table}%
	\caption{Значения коэффициентов параметров ARIMAX(2, 0, 3)x(0, 1, 1, 7) с экзогенными переменными: индикатор дня недели и индикатор Рождества}\label{tbl:cmp-1}
	\centering
	\begin{tabular}{|l|l|c|c|c|c|}
		\hline
			Параметр           &     Значение коэффициента  &  Дисперсия  &  P-value
		\hline
			intercept          &     3.7294                 &     1.686 &   0.027
			weekday_Friday     &     0.0159                 &  2440.530 &   1.000
			weekday_Monday     &     0.0045                 &  5202.932 &   1.000
			weekday_Saturday   &    -0.0011                 &  7994.566 &   1.000
			weekday_Sunday     &     0.0055                 &  4295.359 &   1.000
			weekday_Thursday   &    -0.0069                 &  5951.614 &   1.000
			weekday_Tuesday    &    -0.0007                 &  7381.431 &   1.000
			weekday_Wednesday  &     0.0078                 &  3557.219 &   1.000
			xmas               & -8738.4506                 &   399.352 &   0.000
			ar.L1              &     0.0149                 &     0.067 &   0.824
			ar.L2              &     0.8291                 &     0.056 &   0.000
			ma.L1              &     0.3286                 &     0.074 &   0.000
			ma.L2              &    -0.5438                 &     0.062 &   0.000
			ma.L3              &    -0.1352                 &     0.038 &   0.000
			ma.S.L7            &    -0.9363                 &     0.017 &   0.000
			sigma2             &  1.264e+06                 &  3.87e+04 &   0.000
		\hline
	\end{tabular}
	\label{tbl:arimax_coeffs_exogen}
\end{table}


\section{Построение модели LSTM}

\section{Построение модели FCM}

arima + lstm:
фичей могут быть остатки arima
или само предскзаание arima

но не сработало!

\section{Тестирование разработанной системы}

Для тестирования системы были построены 3 варианта карты:
\begin{itemize}
	\item Карта без связей.
	\item Полносвязная карта.
	\item Карта с осмысленными связями.
\end{itemize}

% todo посмотреть, как модели обучились на тестовых данных

Все карты обучались 100 эпох, имели размерность скрытого состояния 100,
количество исторических данных, обрабатываемых с помощью LSTM на одной итерации -~ 100.
Так же как и количество данных на выходе, сгенерированных.

% todo таблица
% Время обучения
% Качество

Карта без связей \ref{img:lstmfcm_empty}, обученная на тестовых данных
показала отличный результат для $ x_1, x_2, x_3, x_4  $. Однако ошибка
для $ x_5, y $ довольно велика. Такие предсказания эквивалентны предсказаниям
отдельно взятых моделей LSTM, обученных для предсказания временного ряда
по истории только этого временного ряда.

Карта, которая представляет из себя полносвязный граф \ref{img:lstmfcm_fc},
из-за избыточности связи получила предсказания для $ x_1, x_2, x_3 $ даже
хуже, чем карта без связей. Это можно объяснить излишней сложностью модели.
Модель слишком усложнена и не может обучиться за заданное количество эпох.

Карте с осмысленными связями \ref{img:lstmfcm_meaningful} удалось сохранить качество
предсказаний для $ x_1 --- x_4 $ таким же, как и у карты без связей.
А так же удалось получить улучшение предсказания для $ y $.
Неожиданный результат получили для $ x_5 $. todo как так?

Для карт, у которых были связи с $ x_5, y $ получилось лучше предсказать значения
для этих рядов. Потому что значения этих рядов зависят от $ x_1 --- x_4 $.

Кроме того, на долгосрочных предсказаниях видно, что $ x_4 $ имеет проседания.
Вероятно, это связано с ограничениями, рассмотренными во второй главе.
Так как модель обучается на одних данных, а после предсказания, модель возвращает
данные в новой области значений, будущие предсказания становятся все более нестабильными.
todo не понятно, почему их нет на предсказании LSTM

\def\figurename{Рис}
\begin{figure}[t]
	\centering
	\includegraphics[width=0.7\columnwidth]{./img/lstmfcm_empty.png}
	\includegraphics[width=0.9\columnwidth]{./img/lstmfcm_empty_prediction.png}
	\caption{Карта без связей концептов}
	\label{img:lstmfcm_empty}
\end{figure}

\begin{figure}[t]
	\centering
	\includegraphics[width=0.7\columnwidth]{./img/lstmfcm_fc.png}
	\includegraphics[width=0.9\columnwidth]{./img/lstmfcm_fc_prediction.png}
	\caption{Полносвязная карта}
	\label{img:lstmfcm_fc}
\end{figure}

\begin{figure}[t]
	\centering
	\includegraphics[width=0.7\textwidth]{./img/lstmfcm_meaningful.png}
	\includegraphics[width=0.9\textwidth]{./img/lstmfcm_meaningful_prediction.png}
	\caption{Карта с осмысленно расставленными связями}
	\label{img:lstmfcm_meaningful}
\end{figure}


\section{Экспериментальное сравнение разработанной системы с существующими}

Была построена модель, основанная на LSTM,
которая одновременно анализирует все временные ряды тестовых данных.

Результат предсказаний такой модели представлен на рисунке \ref{img:lstm_only_prediction}.
Результат зашумлен по сравнению с предсказаниями карты (todo не могу объяснить).
И в отличие от карты, даже с осмысленными связями, на этом предсказании не наблюдается скачков
между итерациями предсказаний. И $ x_5 $ тоже удалось предсказать относительно успешно.

Данная модель так же, как и карта работала рекурсивно:
на каждой следующей итерации обрабатывала данные, полученные на предыдущей.
Скачок вначале предсказаний, скорее всего, связан с тем, что после обучения,
скрытое состояние сети выставляется заново, так как меняется его размерность.
Возможно, использование скрытого состояния одинаковой размерности во время
тренировки и во время тестирования, может избавить от этой проблемы. Однако
в LSTM NFCM между итерациями тоже сохраняются эти скачки, хотя между итерациями
скрытое состояние не переинициализируется.

\begin{figure}[t]
	\centering
	\includegraphics[width=0.9\textwidth]{./img/lstm_only_prediction.png}
	\caption{LSTM only model prediction}
	\label{img:lstm_only_prediction}
\end{figure}

\section{Влияние гиперпараметров модели на качество предсказаний}

% todo
% Посмотреть, как карта будет самобалансироваться, если сделать прикольную обратную связь.

При уменьшении количества эпох обучения, карта может расходиться, даже карта с осмысленными связями.

Увеличение параметра $ n\_steps\_in $, количество предыдущих значений временных рядов,
а так же размерность скрытого состояния увеличивают количество требуемой для обучения
памяти не более, чем линейно каждый. Конкретный показатель роста будет зависеть от
плотности связей в карте.
Уменьшение этих параметров, позволяет сэкономить память и ускорить обучение,
но страдает качество предсказаний.


\section{Выводы}

В данной главе была реализована система для автоматического когнитивного картирования с использованием
методов нейронных сетей.
Также описаны инструменты, с помощью которых была реализована система.
Описана структура программного обеспечения, проведено тестирование разработанной системы.
На основе тестирования была дана оценка реализованной системе.

% Следует перечислить, какие практические результаты были получены, а именно: какое программное или иное обеспечение было создано. В число результатов могут входить, например, методики тестирования, тестовые примеры (для проверки корректности/оценки характеристик тех или иных алгоритмов) и др. По каждому результату следует сделать вывод, насколько он отличается от известных промышленных аналогов и исследовательских прототипов.

