\chapter*{Заключение}
\addcontentsline{toc}{chapter}{Заключение}

В ходе работы были решены следующие задачи:


\begin{itemize}
	\item Исследованы существующие методы прогнозирования временных рядов для моделирования социально-экономических;
	\item Разработан алгоритм для интеграции нейросетевых моделей с когнитвными картами;
	\item Спроектирована и реализована модель для когнитивного картирования социально-экономической
	системы с использованием нейросетевых методов;
	\item Предобработан и проанализирован тестовый набор данных, на котором были обучены исследуемые модели;
	\item Разработанная модель протестирована на открытых данных социально-экономических систем;
	\item Разработанный алгоритм нейро-нечеткого моделирования был сравнен с существующими решениями: SARIMAX, LSTM;
\end{itemize}

Разработанная модель хорошо подходит для моделирования сложных систем.
При открытии нового магазина, в текущую модель потребуется просто добавить еще
4 концепта (один для магазина, и 3 для каждой категории товара). В то время
как с помощью других исследованных моделей добавление еще одного параметра
привело бы к необходимости их переобучения.
Результирующая ошибка разработанной модели была незначительно больше,
чем для других моделей (LSTM, SARIMX), но модель на основе когнитивной карты
была более интерпретируема.

Для дальнейших исследований можно исследовать интеграцию когнитивной карты
и авторегрессионной модели, например, SARIMAX. Если использовать алгоритмы
автоматической настройки гиперпараметров для SARIMAX модели, эксперт может не тратить
на это время и сосредоточится на проектировании архитектуры модели, а не на деталях.
Однако для более детального изучения модели эксперту стоит исследовать
значения подобранных параметров а также исследовать значения
коэффициентов полученных моделей. Такая модель будет легче в
обучении. Количество параметров будет тоже меньше, чем у FCM-LSTM.

Также в будущем могут быть исследованы рекуррентные связи:
если граф карты будет содержать циклы, то результат вычислений
концепта, обработанный другими концептами будет поступать на вход
самому себе. Это может как привести к взрыву градиентов, так и к
получению более качественных предсказаний за счет того, что рекуррентность
будет заключаться не только в ячейках модели, (концептах, LSTM модели), но и в
глобальной архитектуре модели. Рекуррентные модели проще поддаются Curriculum Learning \cite{bengio2009curriculum, curriculum_taxonomy}.
А рекуррентные модели для решения задачи классификации способны совершать
предсказания, основанные на таксономии \cite{curriculum_taxonomy}.
но эти предсказания можно уточнить в будущем, если провести больше вычислений.
Однако для такой модели, скорее всего,
простое добавление и удаление концептов будет плохо работать. Такая карта будет
менее гибкая для моделирования и исследования следственно-причинных связей.
При удалении или добавлении новых концептов такую карту нужно будет переобучать полностью
сначала (как минимум ту часть, в которой содержится обратная связь).

Еще одним из направлений для дальнейших исследований является то,
как распространяется градиент при обучении модели. В разработанной
модели каждый концепт обучался изолированно относительно других концептов.
Если позволить градиенту во время обучения распространяться на другие
концепты, теоретически, можно повысить качество предсказаний такой модели.
