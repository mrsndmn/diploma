\chapter*{Заключение}
\addcontentsline{toc}{chapter}{Заключение}
% В заключении в тезисной форме необходимо отразить результаты работы:


В ходе работы были решены следующие задачи:

\begin{itemize}
	\item Разработан алгоритм для интеграции нейросетевых моделей с когнитвными картами;
	\item Сравнены разные модели предсказания временных рядов между собой;
	\item Спроектирована и реализована модель для когнитивного картирования с использованием
	авторегрессионных и нейросетевых методов.
	\item Споректировано и реализовано веб-приложение для автоматического когнитивного картирования;
	\item Система протестирована;
\end{itemize}

% В будущем возможно использование гибридных нейро-нечетких систем не только для оптимизации
% нечетких когнитивных карт, но и для уменьшения вычислительных мощностей, требуемых для обучения
% нейросетей за счет использование некоторых экспертных знаний, которые должны отобразиться
% в топологии разрабатываемой нейросети.

% Хотя нейро-нечеткие система в виде TTR NFCM показала себя лучше классической НКК,
% для реализации данного метода все еще нужна работа эксперта: эксперт должен определить
% связи концептов. Хотя на небольших объемах данных это еще возможно, но при увеличении
% количества данных одному человеку их проанализировать становится слишком сложно.
% Поэтому в будущем можно сравнить исследованный метод с другими нейронными или
% статистическими методами для предсказания timeseries данных.

% Для этого также можно использовать XGBoost (бустинг деревьев решений), Long short-term memory (LSTM),
% Random Vector Functional-Link (RVFL), Vector Auto Regression (VAR).
% Плюсом таких методов является то, что при работе с ними, эксперту не требуется определять степень влияния
% одних концептов на другие.

% я не хочу строить карту!
% я хочу залить данные и получить результат
% в кк результат неявный
% и для настройки карты, подбора функции активации нужно
% много мароки