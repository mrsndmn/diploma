\chapter*{Введение}
\label{sec:afterwords}
\addcontentsline{toc}{chapter}{Введение}

Моделирование социально-экономической системы может быть использовано для предсказания
некоторых параметров этой системы в будущем.

Для анализа сложных систем, в которых необходимо принимать к сведению
не только результат прогнозирования,
но и причинно-следственные связи,
используются нечеткие когнитивные карты \cite{osoba2019dags} \cite{kosko1986fuzzy}.
Однако оригинальный алгоритм работы когнитивных карт можно улучшить,
если использовать для оптимизации весов карты нейронные сети.
Такие системы называются нейро-нечеткими.
Такие системы позволяют снизить нагрузку на экспертов и
добиться лучшей точности результата работы системы, но и лучшей интерпретации по сравнению
с нейросетевыми методами, которые не используют экспертные знания.

Актуальность данной работы обусловлена тем, что предиктивные модели
позволяют экспертам лучше понять процессы, которые происходят в
исследуемой системе. Общая идея использования когнитивной карты
как способа структурирования экспертных знаний о системе
может быть использована так же и для задачи классификации.

В аналитической части исследуются существующие методы прогнозирования
временных рядов.

В теоретической части представлена модель нечеткой когнитивной карты с использованием
рекуррентных нейросетей (LSTM).

В инженерном разделе будет спроектирована архитектура модели и
система непрерывной интеграции.

В последнем разделе исследуется набор данных для обучения модели.
Разработанная нейро-нечеткая система будет протестирована и сравнена
с существующими моделями.