\chapter*{Введение}
\label{sec:afterwords}
\addcontentsline{toc}{chapter}{Введение}

Моделирование сложной системы может быть использовано для предсказания
некоторых параметров этой системы в будущем.

Для анализа сложных систем, в которых необходимо принимать к сведению не только результат прогнозирования, но и причинно-следственные связи, используются нечеткие когнитивные карты \cite{osoba2019dags} \cite{kosko1986fuzzy}. Однако оригинальный алгоритм работы когнитивных карт можно улучшить, если использовать для оптимизации весов карты нейронные сети. Такие системы называются нейро-нечеткими. Такие системы позволяют снизить нагрузку на экспертов и добиться лучшей точности результата работы системы.

Актуальность данной работы обусловлена тем, что исследованные на простых
данных свойства нечетких когнитивных карт могут быть обобщены и использованы
для более сложных задач.
Кроме того, в работе используются современные методы оптимизации: для оптимизации весов карты используются нейронные сети.

В теоретической части представлена модель нечеткой когнитивной карты с использованием
нейросетей.

В последнем разделе разработанная нейро-нечеткая система будет протестирована.