\chapter*{Введение}
\label{sec:afterwords}
\addcontentsline{toc}{chapter}{Введение}

Моделирование социально-экономической системы может быть использовано для предсказания
некоторых параметров этой системы в будущем.

Для анализа сложных систем, в которых необходимо принимать к сведению не только результат прогнозирования, но и причинно-следственные связи, используются нечеткие когнитивные карты. Однако оригинальный алгоритм работы когнитивных карт можно улучшить, если использовать для оптимизации весов карты нейронные сети. Такие системы называются нейро-нечеткими. Такие системы позволяют снизить нагрузку на экспертов и добиться лучшей точности результата работы системы.

Актуальность данной работы обусловлена тем, что предиктивные модели
могут быть полезны не только для анализа социально-экономической системы, но и для других задач.
Исследование экономической обстановки в регионах может быть интересно в том числе и органам государственной власти.
Кроме того, в работе используются современные методы оптимизации: для оптимизации весов карты используются нейронные сети.

В теоретической части представлена модель нечеткой когнитивной карты с использованием
нейросетей.

В инженерном разделе будут рассмотрены архитектура веб-приложения и методы обработки данных.

В последнем разделе разработанная нейро-нечеткая система будет протестирована.