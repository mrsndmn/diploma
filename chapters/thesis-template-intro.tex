\chapter*{Введение}
\label{sec:afterwords}
\addcontentsline{toc}{chapter}{Введение}

Моделирование социально-экономической системы может быть использовано для предсказания
некоторых параметров этой системы в будущем.
Для анализа сложных систем, в которых необходимо принимать к сведению
не только результат прогнозирования, но и причинно-следственные связи,
используются нечеткие когнитивные карты \cite{osoba2019dags, kosko1986fuzzy}.
Однако оригинальный алгоритм работы когнитивных карт можно улучшить,
если использовать для оптимизации весов карты нейронные сети.
Такие системы называются нейро-нечеткими.
Такие системы позволяют снизить нагрузку на экспертов и
добиться лучшей точности результата работы системы, но и лучшей интерпретации по сравнению
с нейросетевыми методами, которые не используют экспертные знания.

Понимание процессов, происходящих в социально-экономических системах
может быть полезно как для бизнеса, так и для государства.
Многие организации имеют большое количество статистических данных.
С помощью этих данных можно моделировать поведение
экономической системы, в которой работает данная организация.
Моделирование сложных социально-экономических систем обычно
включает в себя построение прогнозов на будущее.
Однако не все существующие методы позволяют
легко интерпретировать результаты предсказаний.
Актуальность данной работы обусловлена тем, что предиктивные экспертные модели
позволяют лучше понять процессы, которые происходят в
исследуемой системе. Общая идея использования когнитивной карты
как способа структурирования экспертных знаний о системе
может быть использована так же и для других задач машинного обучения.

Для построения прогноза в работе используются рекуррентные нейронные
сети. Нейронные сети --- это современный и постоянно
развивающийся способ моделирования. Количество исследований
на эту тему ежегодно растет.

В аналитической части исследуются существующие методы прогнозирования
временных рядов, сравниваются их характеристики, исследуется
возможность интерпретации этих моделей. Так же рассматривается
модель когнитивных карт. В конце аналитической части определяется
цель данной работы и ставятся задачи.

В теоретической части представлена модель нечеткой когнитивной карты с использованием
рекуррентных нейросетей, а именно LSTM. Исследуются возможные проблемы разработанной модели.

В инженерном разделе будет спроектирована архитектура модели и
система непрерывной интеграции. Рассмотрены процессы работы с построенной моделью.

В последнем разделе исследуется набор данных для обучения модели.
Разработанная нейро-нечеткая система будет протестирована и сравнена
с существующими моделями.